\documentclass{article}
\usepackage{amsmath}
\usepackage{amsfonts}
\usepackage{graphicx}
\usepackage{tcolorbox}

\title{Characterizing Information Value of Energy via an Optimal Control Problem}
\author{Kavin M. Govindarajan}
\date{\today}

\begin{document}

\maketitle

\section{Introduction}
This document defines an optimal control problem (OCP) for generating ergodic trajectories in a solar-powered robot. The robot's state includes its position $(x, y)$ and battery state of charge (SOC). The control input is the robot's velocity $u = (v_x, v_y)$. The battery discharge rate is proportional to $\|u\|$, and the solar charging model follows a diurnal cycle. The goal is to compute an information-optimal trajectory over a finite time horizon $T$, balancing exploration and energy consumption.

We reference Kaleb's Eclares work \cite{naveedEclaresEnergyAwareClarityDriven2024} for the ergodicity formulation.

\section{Preliminaries}
\subsection{Variable Names}
Define the following states:
\begin{itemize}
    \item $x(t)$: position of the robot at time $t$
    \item $u(t)$: control input (velocity) at time $t$
    \item $b(t)$: state of charge at time $t$
    \item $r(t)$: solar irradiance at time $t$
\end{itemize}

Define $T_h$ to be the long-horizon exploration time window, such that
\begin{equation}
    T \leq t \leq T + T_h
\end{equation}

Further define:
\begin{itemize}
    \item $\phi(x)$ as the spatial distribution of information
    \item $p(x,t)$ as the time-averaged trajectory distribution over $T_h$
\end{itemize}

\subsection{Ergodicity}
The ergodicity metric $\mathcal{E}(p, \phi)$ quantifies the mismatch between the trajectory distribution $p(x)$ and the information density $\phi(x)$. The ergodic control problem aims to minimize this metric.

We can measure ergodicity via the spectral norm difference:
\begin{equation}
    \epsilon = \sum_k \lambda_k \left| c_k - \hat{c}_k \right|^2
\end{equation}
where $c_k$ are the Fourier coefficients of $\phi(x)$ (the target information distribution), $\hat{c}_k$ are the Fourier coefficients of $p(x)$ (the actual information distribution), and $\lambda_k$ are weights (Fourier coefficients) for the spatial frequencies.


\begin{center}
\begin{tcolorbox}[colback=gray!30, colframe=white, width=\textwidth, boxrule=0pt, arc=0pt, auto outer arc]
    This ensures that the robot's motion explores the environment in a manner consistent with the desired information distribution.
\end{tcolorbox}
\end{center}

To incorporate energy constraints, we introduce a time-dependent weight:
\begin{equation}
\mathcal{E}_w(p, \phi) = \int_0^T \omega(t) \sum_k \lambda_k \left| c_k - \hat{c}_k \right|^2 dt
\end{equation}
where $\omega(t) = \frac{b(t)}{b_{\max}}$, meaning information gain is prioritized when energy is available.

\subsection{Energy Dynamics}
The energy dynamics of the system are given by:
\begin{equation}
    \dot{b} = - \alpha \|u\|^3 + r(t)
\end{equation}
where $\alpha > 0$ is a scaling factor for velocity-based energy consumption, and $r(t)$ is the solar charging rate, typically sinusoidal or piecewise defined.

The solar irradiance profile, denoted as \( r(t) \), varies throughout the day following a diurnal pattern. A simple model that captures this variation is:

\begin{equation}
    r(t) = r_{\max} \cdot \max\left(0, \sin\left(\frac{\pi (t - t_{\text{sunrise}})}{t_{\text{sunset}} - t_{\text{sunrise}}} \right) \right)
\end{equation}

where:
\begin{itemize}
    \item \( r_{\max} \) is the peak irradiance at solar noon (typically around 1000 W/m²).
    \item \( t_{\text{sunrise}} \) and \( t_{\text{sunset}} \) define the daylight hours.
    \item The sine function ensures that irradiance rises in the morning, peaks at noon, and falls in the evening.
\end{itemize}

This model assumes a clear-sky condition, without atmospheric effects or seasonal variations.

\section{Optimal Control Problem Formulation}
Minimize the ergodic metric:
\begin{equation}
    \min_u \int_0^T \sum_k \lambda_k \left| c_k - \hat{c}_k \right|^2 dt
\end{equation}
subject to the dynamics:
\begin{align}
    \dot{x} &= u \\
    \dot{b} &= - \alpha \|u\|^3 + r(t)
\end{align}
and the constraints:
\begin{align}
    \|u(t)\| &\leq u_{\max} \\
    b(t) &\geq 0 \\
    b(T) &= b_T
\end{align}

\section{Indirect Methods Solution}
We proceed using indirect methods. We first compute the following Hamiltonian:
\begin{equation}
   H = \sum_k \lambda_k \left| c_k - \hat{c}_k \right|^2 + \lambda_x^T u + \lambda_{b} \left( - \alpha \|u\|^3 + r(t) \right)
\end{equation}
where $\lambda_x$ and $\lambda_{b}$ are the costate variables.

This yields the following costate dynamics:
\begin{align}
   \dot{\lambda}_x &= -\frac{\partial H}{\partial x} \\
   \dot{\lambda}_{b} &= -\frac{\partial H}{\partial b}
\end{align}
   
The optimal control $u^*$ must satisfy:
\begin{equation}
   \frac{\partial H}{\partial u} = 0
\end{equation}

and must satisfy the following boundary conditions:
\begin{align}
   x(0) &= x_0 \\
   b(0) &= b_0 \\
   b(T) &= b_T \\
   \lambda_x(T) &= 0 \\
   \lambda_{b}(T) &= 0
\end{align}

Thus, we have derived a two-point boundary value problem. We can use numerical methods such as the shooting method or collocation method to solve the resulting two-point boundary value problem (BVP).

\section{Information Value of Energy}
To compute the information value of energy, define:
\begin{equation}
    \mathcal{I}(T_h, b_T) = \frac{1}{T_h - T} \int_T^{T+T_h} \sum_k \lambda_k \left| c_k - \hat{c}_k \right|^2 dt
\end{equation}
The marginal information value of energy is:
\begin{equation}
    \frac{d}{d (b_T)} \mathcal{I}(T_h, b_T)
\end{equation}
which quantifies the additional information gain per unit SOC.

\subsection{Brute Force Approach}
This essentially sets up an independent OCP for computing the information value of energy. We can solve the same ergodic optimal control formulation from before, but varying the boundary conditions:
\begin{itemize}
    \item No terminal condition on $b$. $b(T) = b_T$ is the only constraint.
    \item Sweep through increasing values of $T_h$
    \item Sweep through various values of $b_T$
\end{itemize}

This will give us a massive surface of information value of energy, from which we can quantify the marginal information value of energy.

If we allow for charging dynamics, I anticipate that there may be some periodic nature that arises for increasing values of $T_h$.

\bibliographystyle{IEEEtran}
\bibliography{references}

\end{document}