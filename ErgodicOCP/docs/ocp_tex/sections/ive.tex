To compute the information value of energy, define:
\begin{equation}
    \mathcal{I}(T_h, b_T) = \frac{1}{T_h - T} \int_T^{T+T_h} \sum_k \lambda_k \left| c_k - \hat{c}_k \right|^2 dt
\end{equation}
The marginal information value of energy is:
\begin{equation}
    \frac{d}{d (b_T)} \mathcal{I}(T_h, b_T)
\end{equation}
which quantifies the additional information gain per unit SOC.

\subsection{Brute Force Approach}
This essentially sets up an independent OCP for computing the information value of energy. We can solve the same ergodic optimal control formulation from before, but varying the boundary conditions:
\begin{itemize}
    \item No terminal condition on $b$. $b(T) = b_T$ is the only constraint.
    \item Sweep through increasing values of $T_h$
    \item Sweep through various values of $b_T$
\end{itemize}

This will give us a massive surface of information value of energy, from which we can quantify the marginal information value of energy.

If we allow for charging dynamics, I anticipate that there may be some periodic nature that arises for increasing values of $T_h$.